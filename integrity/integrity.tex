% 运营商抓取完整性 和 增量抓取服务 文档
\documentclass[UTF8]{ctexart}

\usepackage{minted}


\title{运营商抓取完整性 和 增量抓取服务 文档}
\author{陈雕}


\begin{document}

\maketitle
\newpage

\tableofcontents
\newpage


\section{概述}
按照完整性的维度划分, 可以将抓取完整性划分为 \textbf{空间完整性} 和 \textbf{时间完整性}.

\begin{description}
	\item[时间复杂度] 指的是诸如账单/通话详单/短信详单抓取的月份的完整性
	\item[空间复杂度] 指的是诸如每个月的详单记录数/用户个人信息的各个字段的完整性
\end{description}

抓取完整性一方面可以用来衡量抓取服务的质量, 另一方面还可以用来做增量抓取.  所谓增量抓取, 
即当同一用户多次抓取时, 后面只抓取尚未抓取过的数据, 而复用已经抓取过的数据. 可以设想, 当
已抓取数据足够完整(包括时间和空间数据), 通过增量抓取几乎没有任何抓取代价.

\section{抓取完整性}
当前的运营商抓取完整性服务监测了运营商 抓取账单/通话详单/短息详单 的时间和空间完整性. 相关
信息存储在 MYSQL 数据库中. 存库时只插入, 不更新, 因此该库可以记录每次抓取的完整性数据细节.
相关数据库建表语句为:

\begin{minted}[breaklines=true, fontsize=\tiny]{mysql}
+-----------------+---------------------+------+-----+-------------------+-----------------------------+
| Field           | Type                | Null | Key | Default           | Extra                       |
+-----------------+---------------------+------+-----+-------------------+-----------------------------+
| id              | bigint(20) unsigned | NO   | PRI | NULL              | auto_increment              |
| mo_phonedata_id | bigint(20)          | NO   | MUL | 0                 |                             |
| user_source     | varchar(32)         | NO   | MUL |                   |                             |
| crawler_channel | tinyint(4)          | NO   |     | 0                 |                             |
| session_id      | varchar(64)         | NO   | MUL |                   |                             |
| month           | varchar(10)         | NO   |     |                   |                             |
| bill_type       | tinyint(4)          | NO   |     | 0                 |                             |
| total_records   | int(11)             | NO   |     | 0                 |                             |
| crawled_records | int(11)             | NO   |     | 0                 |                             |
| create_time     | timestamp           | NO   | MUL | CURRENT_TIMESTAMP |                             |
| update_time     | timestamp           | NO   |     | CURRENT_TIMESTAMP | on update CURRENT_TIMESTAMP |
+-----------------+---------------------+------+-----+-------------------+-----------------------------+
\end{minted}

该表需要说明两点:
\begin{enumerate}
	\item session id. 当前的抓取框架还没有发展出 session id, 因此目前使用 uid 代替 session id. 注意到,
使用 session id 可以记录每个 session 的抓取完整性数据. 但是每个抓取 session 的 uid 时相同的, 因此当前
的 uid 并不能代替 session ID 的功能, uid 只是 session ID 的占位符

	\item total records. 该表使用了 total records 和 crawled records 两个字段来记录抓取的空间完整性. 
其中 crawled records 用于记录用户实际抓取到的记录数, total records 用于记录运营商返回的该用户的总记录数.
然而并非所有运营商都会返回该数据, 此时默认 total records = crawled records. 当前各运营商并没有抓取运营商
总记录数, 因此当前总是 total records = crawled records. 我们会在后面的抓取中对此进行改进.
\end{enumerate}


\section{增量抓取}
增量抓取依赖于抓取数据的完整性.

当抓取服务接收到用户的抓取请求时, 会首先查询用户的抓取完整性数据. 以通话记录抓取为例, 设期望抓取的月份为 Em ( Expected Months ),
已抓取的月份为 Cm ( Crawled Months ), 则增量抓取会取两者的差集. 即抓取月份 $$ M = Em - Cm $$.

差集抓取的核心在于抓取完整性的判断. 当前的抓取服务尚难以精确判断一个月抓取数据是否完整. 因此当前采用如下规则进行判断:
\begin{enumerate}
	\item 该月通话记录为 0, 则该月抓取不完整, 需重新抓取;
	\item 该月 total racords != crawled records, 则该月抓取不完整, 需重新抓取;
	\item 当前月数据默认总是不完整, 需要重新抓取;
	\item 用户显式指定不强制更新当前月时, 可以不每次重新抓取当前要详单. 该规则适用于每个月详单均需要短信验证码的运营商.
\end{enumerate}

\end{document}