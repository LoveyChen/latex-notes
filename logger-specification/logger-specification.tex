%% RONG360 logger specification

\documentclass[UTF8,fntef]{ctexart}

%\usepackage{listings}
\usepackage{spverbatim}

\usepackage{hyperref}
%\usepackage{cleveref}

\title {融360日志规范}
\author{陈雕\\
	\href{mailto:chendiao@rong360.com}{chendiao@rong360.com} }


\ctexset{
	section = {
		name = {第,节},
		number = \chinese{section},
	}
}

\begin{document}
   \maketitle
	
\section{概述}
	\subsection{日志}
	日志是为了开发人员查看系统运行情况和追查系统bug而记录的文本. 遵循统一约定规范的, 结构化的日志记录能够以较小的存储成本, 帮助开发人员快速发现和定位系统问题.
	
	日志不仅是面向系统开发人员的, 更是面向日志读者的. 比如如下日志记录:
	
	\begin{spverbatim}
	ERROR: Save failed - SQLException ...
	\end{spverbatim}
	
	对于系统开发人员, 可能该日志已经足以说明问题, 开发人员也可以根据其中的信息找到并修复 bug; 但是对于非该系统的开发人员, 如下的日志记录能够更好的表述系统当前的问题:
	
	\begin{spverbatim}
	ERROR: Save failed - Entity = Person, Data = {id = 123, data="I am the data partion" } - SQLException ...
	\end{spverbatim}
	
	总的来说, 日志具有时间有序的特点, 其内容记录了系统中 \textbf{何时(WHEN, \textit{时间戳})} \textbf{何地(WHERE, \textit{程序位置信息})}  \textbf{何事(WHAT, \textit{Log body})} 发生. 对于系统异常日志, 还需要给出事情发生的 \textbf{原因(WHY, \textit{ Exception stack trace})}. 为了帮助日志分析人员重现问题, 则需要对事件发生 \textbf{过程(HOW)} 进行较为清晰的描述.
	
	日志开源框架, 如 \href{http://logging.apache.org/log4j/}{Log4J}、\href{http://docs.oracle.com/javase/8/docs/technotes/guides/logging/}{Java Logging API}、\href{http://www.slf4j.org/}{SLF4J} 等, 均可通过简单配置, 在日志中打印 WHEN 和 WHERE 信息. 本规范目的是希望通过约定规范日志中的 WHAT (包括 WHY 和 HOW) 信息, 并提供统一的、规范的日志接口, 使系统开发人员能够通过简单的配置和操作, 统一规范的记录日志.



\section{规范}
	\subsection{等级规范}
	根据日志记录内容对系统的影响情况, 本规范允许如下四个日志等级.
	
	\begin{description}
		\item[DEBUG] 调试日志, 主要用于记录RD开发调试时的相关信息, 线上系统不记录该级别日志.
		\item[INFO]	一般消息日志, 主要用于记录系统运行日志, 该级别日志属于系统正常运行时的日志
		\item[WARNNG] 警告日志, 主要用于记录系统运行中的异常行为, 但是该异常行为不影响系统的正常运行
		\item[ERROR] 错误日志, 主要用于记录系统运行中的错误, 这类错误可能会引起系统宕机
	\end{description}
	
	很多同学会在开发过程中使用 System.out.println 的形式记录日志, 以便调试. 但是该方式不能有效的系统记录的日志进行分级, 
	对日志的分析和理解造成不便, 在提交到代码库之前一定要记得删除这类测试代码. 一种备选方案是, 讲此类日志记录为 debug级别,
	因为线上系统不会打印该级别日志, 因此不会对线上系统的日志分析造成困扰.
	
	\subsection{字段规范}
	为了方便定位系统问题, 记录的日志中需要包含如下信息: 
	\begin{spverbatim}
[日期 时间] [日志级别] [程序位置信息] [thread-name:currnt-time][logid] [ip] [uri] [merchant-id] [user-id] [session-id][current-time: ms] log-body
	\end{spverbatim}
	
	日志各字段使用分隔符 \textbf{分隔符} 分离.	
	
	\subsection{日志正文规范}
	为了方便查询问题, 	日志正文格式如下:
	
	\begin{spverbatim}
		[模块信息][流程信息][事件简单描述][事件输入参数(可选)][事件输出参数(可选)][异常堆栈(仅异常事件)]		
	\end{spverbatim}
	
	
	以运营商抓取为例, 
	
	\begin{description}
	\item[模块信息] 指的是记录日志的模块或服务的名称, 其形式为: \textit{北京移动PC端}、\textit{上海移动SHOP端}、\textit{中国移动APP端} 等
	
	\item[流程信息] 值得是记录日志的模块或服务的流程的名字, 其形式为: \textit{登录}、\textit{发送短信验证码}、\textit{抓取通话详单}等
	
	\item[事件简单描述] 是开发人员对该日志事件的简单描述, 如: \textit{登录失败, 服务密码错误}、\textit{发送短信验证码失败} 等
	
	\item[事件输入参数(可选)] 是记录该日志的事件发生时, 提交给程序的参数, 如登录运营商系统的登录参数. 该字段为可选字段
	
	\item[事件输出参数(可选)] 是记录该日志的事件发生时, 程序的输出信息, 如登录运营商系统失败时运营商的返回源码. 该字段为可选字段
	
	\item[异常堆栈(仅异常事件)] 是异常事件发生时的异常堆栈. 该字段仅记录异常日志时打印
	\end{description}
			
	\subsection{接口规范}
	\subsubsection{@logConfig注解}
	当一个类需要使用日志规范时, 需要在定义类时添加 @logConfig 注解.该注解具有参数 module 参数, 该参数标明了该类所属模块.
	
	\subsubsection{@logFlow注解}
	当一个方法需要使用日志规范时, 需要在方法签名处添加 @logFlow 注解. 该注解具有一个参数 flow, 该参数标明了该方法所处的流程, 默认值为方法名.
	
	\subsubsection{在程序中记录日志}
	在程序中使用 \spverb!log.info()! 记录日志.
	
	\begin{spverbatim}
	
	\end{spverbatim}	
	
	接口规范将在 章节~\ref{sec:implementation} 中详细介绍.
	
	
	

\section{实现}\label{sec:implementation}
为了规范化日志, 本规范实现了统一的日志记录接口.


	
	
\appendix
\section{常见日志场景说明}
	\begin{itemize}
		\item \textbf{关键函数入口日志} \\
		为了方便查看函数的调用情况, 可以在函数入口处打印函数入口日志, 记录为 INFO 级别日志, 需要打印出函数名和传入的实参的值, 如:
		
		\item \textbf{非关键函数入口日志} \\
		不建议记录非关键函数的入口日志, 如确因调试原因需要打印该类型日志,  记录为 DEBUG 级别日志, 需要打印出函数名和传入的实参值:
		
		\item \textbf{系统关键信息日志} \\
		以运营商抓取中的场景为例, 如 判断短信验证码正确与否时, 短信验证码正确有唯一的识别标识( the-only-matched-condition ), 
		则当验证码不匹配时, 记录级别为 WARN 的验证码不匹配日志, 其内容包括用户输入的验证码和运营商系统返回的源码:
		
		如果验证码正确没有较为明显的识别标识, 则可以明确表示验证码不匹配( the-not-matched-condition ), 
		则不匹配情形记录 WARN 级别的验证码不匹配日志, 其内容包括用户输入的验证码和运营商系统返回的源码; 并对其它情形记录 INFO 级别
		的日志, 其内容包括 运营商系统返回的源码:
		
		\item \textbf{系统异常日志} \\
		当系统执行捕获到异常时, 需要打印系统异常日志, 该类型日志记录为 WARN 级别, 需要打印出异常的具体调用栈信息, 如因数据问题引起
		的系统异常, 需要同时打印出引起异常的原始数据:
		
		\item \textbf{错误日志} \\
		当系统运行时需要无法修复的错误, 需要重启系统时, 需要记录错误日志, 日志级别为 ERROR, 需要打印出该错误的具体原因:
		
	\end{itemize}
\end{document}