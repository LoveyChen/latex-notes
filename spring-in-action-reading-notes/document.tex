% Spring in action, 4th edition, reading notes

\documentclass[UTF8, heading=true, scheme=chinese]{ctexart}

\usepackage{underscore}
\usepackage{minted}
\usepackage{hyperref}

\usepackage{float}
\floatstyle{boxed}
\restylefloat{figure}

\title{Spring in action, 4th edition 读书笔记}
\author{陈雕}

\begin{document}
\maketitle
\newpage

\tableofcontents
\newpage


\section[Advanced wiring]{Advanced wiring}
本章介绍了 wiring 的一些高级技巧, 包括:
\begin{enumerate}
	\item[\nameref{sec:profiles}] 主要介绍了 profiles 的使用, 包括如何定义 profiled beans 以及 如何在应用中启用相应的 profile
	\item[\nameref{sec:conditional}] 主要介绍了 @Conditional 注解, 以及如何定义和使用 Conditional beans
	\item[\nameref{sec:qualifiers}] 主要介绍了 autowiring 时多个beans满足注入条件时的处理办法
	\item[\nameref{sec:scope}] 介绍了Spring中 beans 的 scope 的概念, 以及设置 beans 的scope的条件和方法
	\item[\nameref{sec:runtime}] 主要介绍了如何在定义 beans 使用外部值(如属性文件中定义的值) 以及更为通用的 Spring Expression Language (SpEL)
\end{enumerate}


\subsection[Environment and Profiles]{Environment and Profiles} \label{sec:profiles}
一个应用, 从开发到测试到部署, 通常会在不同的环境中运行, 而不同环境可能有不同的应用配置, 如开发/测试和运行环境的数据库配置可能就各部相同.
一个可能的解决办法是分别为各个环境写不同的配置文件, 在不同环境下面, 使用各自的配置文件替换应用的配置文件即可.

然而, 一旦 Spring 的配置文件发生改变(无论是隐式配置还是显示的Java或XML配置), 就涉及到应用的重新构建和部署, 也就带来QA或OP在应用中引入bug的风险.
一种更好的方式是使用 Spring 的 profile.

Profile 由 Spring 3.1 引入. 主要使用方法是 使用 @Profile 注解(Java配置时) 或 使用 beans 标记的 profile 属性.
\begin{minted}[breaklines=true, showspaces=false, showtabs=false]{java}
@Configuration
public class DataSourceConfig {

	@Bean(destroyMethod="shutdown")
	@Profile("dev")
	public DataSource embeddedDataSource() {
		return new EmbeddedDatabaseBuilder()
		.setType(EmbeddedDatabaseType.H2)
		.addScript("classpath:schema.sql")
		.addScript("classpath:test-data.sql")
		.build();
	}
		
	@Bean
	@Profile("prod")
	public DataSource jndiDataSource() {
		JndiObjectFactoryBean jndiObjectFactoryBean =
		new JndiObjectFactoryBean();
		jndiObjectFactoryBean.setJndiName("jdbc/myDS");
		jndiObjectFactoryBean.setResourceRef(true);
		jndiObjectFactoryBean.setProxyInterface(javax.sql.DataSource.class);
		return (DataSource) jndiObjectFactoryBean.getObject();
	}
}
\end{minted}


\begin{minted}[breaklines=true, showspaces=false, showtabs=false]{xml}
<?xml version="1.0" encoding="UTF-8"?>
<beans xmlns="http://www.springframework.org/schema/beans"
			xmlns:xsi="http://www.w3.org/2001/XMLSchema-instance"
			xmlns:jdbc="http://www.springframework.org/schema/jdbc"
			xsi:schemaLocation="
			http://www.springframework.org/schema/jdbc
			http://www.springframework.org/schema/jdbc/spring-jdbc.xsd
			http://www.springframework.org/schema/beans
			http://www.springframework.org/schema/beans/spring-beans.xsd"
			profile="dev">
			
	<jdbc:embedded-database id="dataSource">
		<jdbc:script location="classpath:schema.sql" />
		<jdbc:script location="classpath:test-data.sql" />
	</jdbc:embedded-database>
</beans>
\end{minted}

如何启用对应的 profile 呢? 主要依靠两个属性 \mintinline{java}|spring.profiles.active| 和 \mintinline{java}|spring.profiles.active|, 其中前者具有更高的优先级.
那么应用如何设置这两个参数呢? 主要可以有如下几种方法:
\begin{itemize}
	\item 设置为 DispatcherServlet 的初始化参数, 参考 web.xml 的相关说明
	\item 设置为 web 应用的上下文参数, 参考 web.xml 的相关说明
	\item 设置为 JNDI 条目
	\item 设置为环境变量
	\item 设置为 JVM 系统参数
	\item 给集成测试类添加 @ActiveProfiles 注解, 制定对应的 profile
\end{itemize}






\subsection[Conditional beans]{Conditional beans} \label{sec:conditional}
事实上, \nameref{sec:profiles} 可以看做时 \nameref{sec:conditional} 的特例. Conditional beans 主要用于在系统满足一定条件时才创建对应的 beans. 
Conditional beans 主要靠 @Conditional 注解 和 @Conditional 制定的规则类来实现,  如
\begin{minted}[breaklines=true]{java}
@Bean
@Conditional(MagicExistsCondition.class)
public MagicBean magicBean() {
	return new MagicBean();
}
\end{minted}

其中, @Conditional 的参数是接口 @Condition 接口的实现类.
\begin{minted}[breaklines=true]{java}
public interface Condition {
	boolean matches(ConditionContext ctxt, AnnotatedTypeMetadata metadata);
}
\end{minted}
当指定的 @Condition 实现类的 matches 返回 true, 则会创建对应的 beans.



\subsection[Primary and Qualifiers]{Primary and Qualifiers} \label{sec:qualifiers}
@Primary 和 @Qualifier 用于解决多个bean都满足注入条件时的歧义问题. 当不使用 @Primary 或 @Qualify  时,
如果有多个bean同时满足注入条件, 则Spring会抛出 \mintinline{java}|org.springframework.beans.factory.NoUniqueBeanDefinitionException|.

此时可以使用 @Primary 注解, 该注解可以在定义有歧义bean的类签名处使用, 也可以在配置时使用. XML配置文件中则使用 beans 的 primary="true" 属性.
\begin{minted}[breaklines=true]{java}
@Component
@Primary
public class IceCream implements Dessert { ... }
\end{minted}

\begin{minted}[breaklines=true]{java}
@Bean
@Primary
public Dessert iceCream() {
	return new IceCream();
}
\end{minted}

显然, 如果多个有歧义的bean同时设置了 @Primary 注解, 则不能有效解决歧义问题. 此时可以使用 qualifier. @Qualifier 注解主要是beans 注入时使用, 即在Autowired注解处使用.
\begin{minted}[breaklines=true]{java}
@Autowired
@Qualifier("iceCream")
public void setDessert(Dessert dessert) {
	this.dessert = dessert;
}
\end{minted}




\subsection[Scope beans]{Scope beans} \label{sec:scope}
默认情况下, beans 被创建为单例模式( Singleton), 即无论一个制定ID的bean被注入到其它beans中时, 注入的总是同一个bean. 

当Singleton不能满足条件时, 还可以指定beans的Scope, Spring定义了如下几种Scope:
\begin{enumerate}
	\item[Singleton] bean被创建为单例模式, 即每次都是同一个bean
	\item[Prototype] 每次注入bean, 都创建一个全新的bean
	\item[Session] 同一个会话中注入的bean总是同一个bean(Web Application) 
	\item[Request] 同一个请求中注入的bean总是同一个bean (Web Application)
\end{enumerate}

对于前两种Scope, 可以在配置Beans使用@Scope指定bean的Scope, 如
\begin{minted}[breaklines=true]{java}
@Bean
@Scope(ConfigurableBeanFactory.SCOPE_PROTOTYPE)
public Notepad notepad() {
	return new Notepad();
}
\end{minted}

而在Web Application中, 则应使用如下方式指定bean的Scope,
\begin{minted}[breaklines=true]{java}
@Component
@Scope(
	value=WebApplicationContext.SCOPE_SESSION,
	proxyMode=ScopedProxyMode.INTERFACES)
public ShoppingCart cart() { ... }
\end{minted}

注意Scope的proxyMode属性. 




\subsection[Scope beans]{Runtime value injection} \label{sec:runtime}
可以在指定bean属性时, 引用外部变量,  如下:
\begin{minted}[breaklines=true]{java}
@Configuration
@PropertySource("classpath:/com/soundsystem/app.properties")
public class ExpressiveConfig {
	@Autowired
	Environment env;
	@Bean
	public BlankDisc disc() {
		return new BlankDisc(
		env.getProperty("disc.title"),
		env.getProperty("disc.artist"));
	}
}
\end{minted}

而使用 Spring Expression Language (SpEL)则具有更好的灵活性.
SpEL的形式是 \#\{...\}, 其表达式可以是
\begin{enumerate}
	\item[字面常量, literal value]	如 \#\{3.14159\}, \#\{'Hello World'\}
	\item[bean, 及其属性或方法]	如 \#\{sgtPeppers\}, \#\{sgtPeppers.artist\}, \#\{sgtPeppers.artist?.toUpperCase()\}
	\item[类的静态成员或方法]	如 \#\{T(java.lang.Math).PI\}, \#\{T(java.lang.Math).random()\}
	\item[算式] 	如 \#\{2 * T(java.lang.Math).PI * circle.radius\}
	\item[正则表达式]	如 \#\{admin.email matches '[a-zA-Z0-9._\%+-]+@[a-zA-Z0-9.-]+\\.com'\}, 结果是 true 或 false
	\item[集合表达式]	如 \#\{jukebox.songs[4].title\}, \#\{jukebox.songs.?[artist eq 'Aerosmith']\},  \#\{jukebox.songs.\^[artist eq 'Aerosmith']\}, \#\{jukebox.songs.![title]\}
\end{enumerate}

注意bean属性中的 ?. 运算符, \#\{sgtPeppers.artist?.toUpperCase()\}, 当 sgtPeppers.artist 为 null, 直接返回 null, 而不用执行 toUpperCase 了, 类似的有集合表达式中 .?[] 操作符.
集合表达式的 .\^[] 和 .![] 则分别取满足条件的第一个和最后一个元素. 
\end{document}